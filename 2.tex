% Options for packages loaded elsewhere
% Options for packages loaded elsewhere
\PassOptionsToPackage{unicode}{hyperref}
\PassOptionsToPackage{hyphens}{url}
\PassOptionsToPackage{dvipsnames,svgnames,x11names}{xcolor}
%
\documentclass[
  11pt,
]{article}
\usepackage{xcolor}
\usepackage[margin=1in]{geometry}
\usepackage{amsmath,amssymb}
\setcounter{secnumdepth}{5}
\usepackage{iftex}
\ifPDFTeX
  \usepackage[T1]{fontenc}
  \usepackage[utf8]{inputenc}
  \usepackage{textcomp} % provide euro and other symbols
\else % if luatex or xetex
  \usepackage{unicode-math} % this also loads fontspec
  \defaultfontfeatures{Scale=MatchLowercase}
  \defaultfontfeatures[\rmfamily]{Ligatures=TeX,Scale=1}
\fi
\usepackage{lmodern}
\ifPDFTeX\else
  % xetex/luatex font selection
\fi
% Use upquote if available, for straight quotes in verbatim environments
\IfFileExists{upquote.sty}{\usepackage{upquote}}{}
\IfFileExists{microtype.sty}{% use microtype if available
  \usepackage[]{microtype}
  \UseMicrotypeSet[protrusion]{basicmath} % disable protrusion for tt fonts
}{}
\usepackage{setspace}
\makeatletter
\@ifundefined{KOMAClassName}{% if non-KOMA class
  \IfFileExists{parskip.sty}{%
    \usepackage{parskip}
  }{% else
    \setlength{\parindent}{0pt}
    \setlength{\parskip}{6pt plus 2pt minus 1pt}}
}{% if KOMA class
  \KOMAoptions{parskip=half}}
\makeatother
% Make \paragraph and \subparagraph free-standing
\makeatletter
\ifx\paragraph\undefined\else
  \let\oldparagraph\paragraph
  \renewcommand{\paragraph}{
    \@ifstar
      \xxxParagraphStar
      \xxxParagraphNoStar
  }
  \newcommand{\xxxParagraphStar}[1]{\oldparagraph*{#1}\mbox{}}
  \newcommand{\xxxParagraphNoStar}[1]{\oldparagraph{#1}\mbox{}}
\fi
\ifx\subparagraph\undefined\else
  \let\oldsubparagraph\subparagraph
  \renewcommand{\subparagraph}{
    \@ifstar
      \xxxSubParagraphStar
      \xxxSubParagraphNoStar
  }
  \newcommand{\xxxSubParagraphStar}[1]{\oldsubparagraph*{#1}\mbox{}}
  \newcommand{\xxxSubParagraphNoStar}[1]{\oldsubparagraph{#1}\mbox{}}
\fi
\makeatother


\usepackage{longtable,booktabs,array}
\usepackage{calc} % for calculating minipage widths
% Correct order of tables after \paragraph or \subparagraph
\usepackage{etoolbox}
\makeatletter
\patchcmd\longtable{\par}{\if@noskipsec\mbox{}\fi\par}{}{}
\makeatother
% Allow footnotes in longtable head/foot
\IfFileExists{footnotehyper.sty}{\usepackage{footnotehyper}}{\usepackage{footnote}}
\makesavenoteenv{longtable}
\usepackage{graphicx}
\makeatletter
\newsavebox\pandoc@box
\newcommand*\pandocbounded[1]{% scales image to fit in text height/width
  \sbox\pandoc@box{#1}%
  \Gscale@div\@tempa{\textheight}{\dimexpr\ht\pandoc@box+\dp\pandoc@box\relax}%
  \Gscale@div\@tempb{\linewidth}{\wd\pandoc@box}%
  \ifdim\@tempb\p@<\@tempa\p@\let\@tempa\@tempb\fi% select the smaller of both
  \ifdim\@tempa\p@<\p@\scalebox{\@tempa}{\usebox\pandoc@box}%
  \else\usebox{\pandoc@box}%
  \fi%
}
% Set default figure placement to htbp
\def\fps@figure{htbp}
\makeatother





\setlength{\emergencystretch}{3em} % prevent overfull lines

\providecommand{\tightlist}{%
  \setlength{\itemsep}{0pt}\setlength{\parskip}{0pt}}



 
\usepackage[]{biblatex}
\addbibresource{references.bib}


\usepackage{hyperref}
\hypersetup{
  colorlinks=true,
  linkcolor=blue,
  urlcolor=blue,
  breaklinks=true,
  pdfborder={0 0 0}
}
\makeatletter
\@ifpackageloaded{caption}{}{\usepackage{caption}}
\AtBeginDocument{%
\ifdefined\contentsname
  \renewcommand*\contentsname{Table of contents}
\else
  \newcommand\contentsname{Table of contents}
\fi
\ifdefined\listfigurename
  \renewcommand*\listfigurename{List of Figures}
\else
  \newcommand\listfigurename{List of Figures}
\fi
\ifdefined\listtablename
  \renewcommand*\listtablename{List of Tables}
\else
  \newcommand\listtablename{List of Tables}
\fi
\ifdefined\figurename
  \renewcommand*\figurename{Figure}
\else
  \newcommand\figurename{Figure}
\fi
\ifdefined\tablename
  \renewcommand*\tablename{Table}
\else
  \newcommand\tablename{Table}
\fi
}
\@ifpackageloaded{float}{}{\usepackage{float}}
\floatstyle{ruled}
\@ifundefined{c@chapter}{\newfloat{codelisting}{h}{lop}}{\newfloat{codelisting}{h}{lop}[chapter]}
\floatname{codelisting}{Listing}
\newcommand*\listoflistings{\listof{codelisting}{List of Listings}}
\makeatother
\makeatletter
\makeatother
\makeatletter
\@ifpackageloaded{caption}{}{\usepackage{caption}}
\@ifpackageloaded{subcaption}{}{\usepackage{subcaption}}
\makeatother
\usepackage{bookmark}
\IfFileExists{xurl.sty}{\usepackage{xurl}}{} % add URL line breaks if available
\urlstyle{same}
\hypersetup{
  pdftitle={William's Update},
  pdfauthor={William Clinton Co},
  colorlinks=true,
  linkcolor={blue},
  filecolor={Maroon},
  citecolor={blue},
  urlcolor={blue},
  pdfcreator={LaTeX via pandoc}}


\title{William's Update}
\usepackage{etoolbox}
\makeatletter
\providecommand{\subtitle}[1]{% add subtitle to \maketitle
  \apptocmd{\@title}{\par {\large #1 \par}}{}{}
}
\makeatother
\subtitle{COMM 498}
\author{William Clinton Co}
\date{August 8, 2025}
\begin{document}
\maketitle
\begin{abstract}
This document presents updated materials based on William and Ilan's
meeting on August 1, 2025 and provides updates and recommendations for
the upcoming course COMM 498.
\end{abstract}

\renewcommand*\contentsname{Table of contents}
{
\hypersetup{linkcolor=}
\setcounter{tocdepth}{3}
\tableofcontents
}

\setstretch{1}
\section{Exercise Recommendations}\label{exercise-recommendations}

\subsection{Class 2: Globalization}\label{class-2-globalization}

\textbf{End-of-class exercise question:}

\begin{itemize}
\tightlist
\item
  Given the benefits of globalization, why is ``slowbalization''
  occurring?
\item
  Crisis reposnce activity
\end{itemize}

\subsection{Class 3: Slowbalization}\label{class-3-slowbalization}

\textbf{End-of-class exercise question:}

\begin{itemize}
\tightlist
\item
  What is your prediction for the global economy in light of
  slowbalization? Will it continue or reverse? Explain your reasoning.
  ask camille about my mtorocycel inquiry \#\# Class 5
\end{itemize}

Turn international market analysis into a competitive activity where
teams act as firms choosing where to expand. Each country option has
different risk-reward profiles (tariffs, cultural barriers, supply chain
complexity), and teams justify their choices. Makes theoretical
frameworks more tangible. Acts as a low-stakes preliminary activity for
final project - gets students acquainted with this, may assist them in
ideation or execution of this project.

\textbf{\emph{Class 5: Cross-National Trade Simulation Activity}} To
explore international trade and comparative advantage hands-on, students
will work in small groups, each representing a different country.

\begin{itemize}
\tightlist
\item
  Group 1: United States\\
\item
  Group 2: China\\
\item
  Group 3: Germany\\
\item
  Group 4: Brazil
\end{itemize}

Objective: Using production data for two goods (e.g.~cars and bananas),
each group will:

\begin{enumerate}
\def\labelenumi{\arabic{enumi}.}
\tightlist
\item
  Calculate their opportunity cost for producing each good.
\item
  Identify their country's comparative advantage based on opportunity
  cost.
\item
  Negotiate trade agreements with other countries to maximize mutual
  benefits using knowledge gained from the course material
\item
  Strategically specialize in the good they produce most efficiently.
\item
  Achieve the highest combined total output via smart specialization and
  trade.
\end{enumerate}

\begin{quote}
This is a \textbf{timed competition:} the group that secures the most
advantageous trade deals \textgreater and maximizes net gains from trade
will be recognized as the top global trading nation.
\end{quote}

Guidelines: - Use basic economic concepts (comparative advantage,
opportunity cost, specialization, gains from trade) - Each trade must be
recorded, including what was traded, with whom, and under what terms -
You must be able to explain the rationale for your decisions and how
opportunity costs informed your strategy

Example: One team represents USA and the other Brazil

\begin{longtable}[]{@{}lll@{}}
\toprule\noalign{}
Country & Cars & Bananas \\
\midrule\noalign{}
\endhead
\bottomrule\noalign{}
\endlastfoot
USA & 10 & 20 \\
Brazil & 5 & 25 \\
\end{longtable}

To determine the winner:

\begin{enumerate}
\def\labelenumi{\arabic{enumi}.}
\tightlist
\item
  Track all trades: Each group must keep a clear record of:

  \begin{itemize}
  \tightlist
  \item
    Goods produced
  \item
    Goods traded (and with whom)
  \item
    Goods consumed after trade
  \end{itemize}
\item
  Calculate total benefit:

  \begin{itemize}
  \tightlist
  \item
    Assign each good a value (e.g., Cars = 10 units, Bananas = 5 units)
  \item
    Sum the total value of the goods each country ends up with
    post-trade
  \item
    Assign a value to each good: Class can agree on a simple point
    system. For example:
  \end{itemize}
\end{enumerate}

1 Car = 10 points\\
1 Banana = 5 points

Now multiply:

USA's total value = (7 Cars × 10) + (15 Bananas × 5) = 70 + 75 = 145
points (if they chose to trade 3 cars)\\
Brazil's total value = (3 Cars × 10) + (10 Bananas × 5) = 30 + 50 = 80
points (if they chose to trade 15 bananas)

\begin{enumerate}
\def\labelenumi{\arabic{enumi}.}
\setcounter{enumi}{2}
\tightlist
\item
  Subtract opportunity costs:

  \begin{itemize}
  \tightlist
  \item
    Use your group's production data to estimate how many units of the
    other good you gave up (opportunity cost) when producing
  \item
    Deduct this from your final value to calculate net benefit (Net
    Benefit = Final Goods Value -- Opportunity Cost Value)
  \end{itemize}
\end{enumerate}

\subsection{Class 4/6/16/17 Current Threats to Globalization/Trade
Institutions/ Global \& Regional Strategies/Technology
Disruption}\label{class-461617-current-threats-to-globalizationtrade-institutions-global-regional-strategiestechnology-disruption}

\begin{itemize}
\tightlist
\item
  Already contains sufficient exercises
\end{itemize}

\subsection{Class 8/9 Plans for Internationalization/ Location
Decision}\label{class-89-plans-for-internationalization-location-decision}

\begin{itemize}
\tightlist
\item
  Add exercises on Ghemewat's CAGE Framework
\item
  Add exercises on Location Grid Evaluation
\end{itemize}

\subsection{Class 13/14 FDI \& M\&A}\label{class-1314-fdi-ma}

\textbf{End-of-class exercise question:}

\begin{itemize}
\tightlist
\item
  Could Daimler Chrysler Failures be avoided? How so? What is your
  recommendations?
\end{itemize}

\subsection{Class 17}\label{class-17}

\begin{itemize}
\tightlist
\item
  AI game
\end{itemize}

\section{William and Camille's Exercise
Recommendations}\label{william-and-camilles-exercise-recommendations}

\subsubsection{Interactive trade or market entry
game}\label{interactive-trade-or-market-entry-game}

Turn international market analysis into a competitive activity where
teams act as firms choosing where to expand. Each country option has
different risk-reward profiles (tariffs, cultural barriers, supply chain
complexity), and teams justify their choices. Makes theoretical
frameworks more tangible. Acts as a low-stakes preliminary activity for
final project - gets students acquainted with this, may assist them in
ideation or execution of this project.

\subsubsection{International crisis response
simulation}\label{international-crisis-response-simulation}

Present a scenario involving a sudden geopolitical, economic, or
health-related disruption. Students take on roles (e.g., regional
director, risk analyst, COO) and collaborate to draft a real-time global
adaptation strategy (encourages fast thinking and practical
problem-solving.

\subsubsection{Final Project - International Expansion
Case}\label{final-project---international-expansion-case}

Objective: Apply strategic analysis tools to identify and solve a
company's core business challenges within an international context,
resulting in a recommendation that addresses both firm-specific and
global market dynamics

\subsubsection{Notes \& suggestions:}\label{notes-suggestions}

\begin{itemize}
\tightlist
\item
  Rubric could include section on presentation of case report (e.g.,
  visuals, organization of information, logical flow of report).
\item
  Could include contingency analysis: missing a risk assessment and plan
  to mitigate potential challenges.
\item
  Important to consider how implementation would vary under conditions
  of different nations (political situation, tastes, social \& economic
  factors)
\item
  Could include contingency analysis: missing a risk assessment and plan
  to mitigate potential challenges.
\item
  Important to consider how implementation would vary under conditions
  of different nations (political situation, tastes, social \& economic
  factors)
\end{itemize}

\subsection{Roleplay}\label{roleplay}

Students have now learned about how firms act when engaging in IB; they
should apply and test this knowledge in role-play situations. The
professor evaluates how effective their choices are and whether this
accurately reflects real decisions.

\subsubsection{Example}\label{example}

\emph{Class 5: Cross-National Trade Simulation Activity} To explore
international trade and comparative advantage hands-on, students will
work in small groups, each representing a different country.

\begin{itemize}
\tightlist
\item
  Group 1: United States\\
\item
  Group 2: China\\
\item
  Group 3: Germany\\
\item
  Group 4: Brazil
\end{itemize}

Objective: Using production data for two goods (e.g.~cars and bananas),
each group will:

\begin{enumerate}
\def\labelenumi{\arabic{enumi}.}
\tightlist
\item
  Calculate their opportunity cost for producing each good.
\item
  Identify their country's comparative advantage based on opportunity
  cost.
\item
  Negotiate trade agreements with other countries to maximize mutual
  benefits using knowledge gained from the course material
\item
  Strategically specialize in the good they produce most efficiently.
\item
  Achieve the highest combined total output via smart specialization and
  trade.
\end{enumerate}

\begin{quote}
This is a \textbf{timed competition:} the group that secures the most
advantageous trade deals \textgreater and maximizes net gains from trade
will be recognized as the top global trading nation.
\end{quote}

Guidelines: - Use basic economic concepts (comparative advantage,
opportunity cost, specialization, gains from trade) - Each trade must be
recorded, including what was traded, with whom, and under what terms -
You must be able to explain the rationale for your decisions and how
opportunity costs informed your strategy

Example: One team represents USA and the other Brazil

\begin{longtable}[]{@{}lll@{}}
\toprule\noalign{}
Country & Cars & Bananas \\
\midrule\noalign{}
\endhead
\bottomrule\noalign{}
\endlastfoot
USA & 10 & 20 \\
Brazil & 5 & 25 \\
\end{longtable}

To determine the winner:

\begin{enumerate}
\def\labelenumi{\arabic{enumi}.}
\tightlist
\item
  Track all trades: Each group must keep a clear record of:

  \begin{itemize}
  \tightlist
  \item
    Goods produced
  \item
    Goods traded (and with whom)
  \item
    Goods consumed after trade
  \end{itemize}
\item
  Calculate total benefit:

  \begin{itemize}
  \tightlist
  \item
    Assign each good a value (e.g., Cars = 10 units, Bananas = 5 units)
  \item
    Sum the total value of the goods each country ends up with
    post-trade
  \item
    Assign a value to each good: Class can agree on a simple point
    system. For example:
  \end{itemize}
\end{enumerate}

1 Car = 10 points\\
1 Banana = 5 points

Now multiply:

USA's total value = (7 Cars × 10) + (15 Bananas × 5) = 70 + 75 = 145
points (if they chose to trade 3 cars)\\
Brazil's total value = (3 Cars × 10) + (10 Bananas × 5) = 30 + 50 = 80
points (if they chose to trade 15 bananas)

\begin{enumerate}
\def\labelenumi{\arabic{enumi}.}
\setcounter{enumi}{2}
\tightlist
\item
  Subtract opportunity costs:

  \begin{itemize}
  \tightlist
  \item
    Use your group's production data to estimate how many units of the
    other good you gave up (opportunity cost) when producing
  \item
    Deduct this from your final value to calculate net benefit (Net
    Benefit = Final Goods Value -- Opportunity Cost Value)
  \end{itemize}
\end{enumerate}

\paragraph{1.1.2 International crisis response
simulation}\label{international-crisis-response-simulation-1}

Present a scenario involving a sudden geopolitical, economic, or
health-related disruption. Students take on roles (e.g., regional
director, risk analyst, COO) and collaborate to draft a real-time global
adaptation strategy (encourages fast thinking and practical
problem-solving).

\paragraph{Example}\label{example-1}

\emph{Class 2: Crisis Response Role-play Activity} To apply strategic
thinking under uncertainty and crisis shocks in international business,
students will engage in a role-play scenario where groups of students
will assume the roles of executives and plan their course of action.

\begin{itemize}
\tightlist
\item
  Group 1: CEO
\item
  Group 2: Regional Director
\item
  Group 3: Global Financial Manager
\item
  Group 4: International Business Development Manager
\end{itemize}

Objective: In groups of 3-4, students will analyze a case in which their
business is under threat of a global crisis (supply chain or
finance-related). They will then design a plan to mitigate the impacts
of this issue and maintain international operations, aligning with their
respective executive role.

Scenario example: Your company is a mid-sized European retail brand
expanding aggressively into Southeast Asia, Latin America, and Africa.
You rely on an AI-based global logistics platform headquartered in
Singapore to manage inventory flows, delivery coordination, and supplier
tracking.

Overnight, the system is compromised by a malware attack traced to a
state-linked hacking group in Russia. Your AI platform is frozen,
shipments are delayed, and customer data may be exposed. The breach puts
your company's supply chain, public image, and financial stability at
risk.

Simulation Flow:

\begin{enumerate}
\def\labelenumi{\arabic{enumi}.}
\tightlist
\item
  10 minutes - Read the scenario and discuss within your group
\item
  15 minutes - Develop a strategic response from your role's perspective
\item
  5 minutes per group - Present your proposed response to the class
\item
  Class-wide debrief - Reflect on alignment/conflict between roles and
  real-world relevance
\end{enumerate}

\paragraph{AI-Centred Activity}\label{ai-centred-activity}

\emph{Class 17: AI and Globalization Simulation Activity} To explore how
artificial intelligence (AI) influences globalization, students will
work in small groups, each representing a distinct domain, sector, or
industry affected by AI in the global economy.

Group Assignments: - Group 1: Supply Chain and Logistics\\
- Group 2: Finance and Banking\\
- Group 3: Manufacturing and Automation

Objective: Using real-world examples and economic concepts discussed in
class, each group will:

\begin{enumerate}
\def\labelenumi{\arabic{enumi}.}
\tightlist
\item
  Investigate how AI is transforming their assigned sector on a global
  scale.
\item
  Identify 2--3 key ways AI has influenced labour, trade, or innovation
  across country borders
\item
  Present a short ``case'' or simulated scenario that demonstrates how
  AI drives globalization in their sector.
\item
  Explain the geopolitical, ethical, and economic implications of these
  changes.
\item
  Compare AI's impact across sectors in a class-wide discussion on
  convergence/divergence in global outcomes.
\end{enumerate}

To Determine the Most Effective Group:

\begin{enumerate}
\def\labelenumi{\arabic{enumi}.}
\tightlist
\item
  \textbf{Content Depth}

  \begin{itemize}
  \tightlist
  \item
    Did the group provide specific, well-researched examples?\\
  \item
    Were connections to globalization clearly made?
  \end{itemize}
\item
  \textbf{Application of Course Concepts}

  \begin{itemize}
  \tightlist
  \item
    Did the group reference key ideas such as technological diffusion,
    labour shifts, or ethical challenges?
  \end{itemize}
\item
  \textbf{Creativity and Clarity}

  \begin{itemize}
  \tightlist
  \item
    Was the scenario engaging and easy to follow?\\
  \item
    Did the group make the topic relevant to real-world IB challenges?
  \end{itemize}
\item
  \textbf{Discussion Facilitation}

  \begin{itemize}
  \tightlist
  \item
    Did the group spark thoughtful discussion in the class-wide
    comparison?
  \end{itemize}
\end{enumerate}


\printbibliography



\end{document}
